\documentclass[12pt, letterpaper]{article}
\usepackage[utf8]{inputenc}
\usepackage{graphicx}
\usepackage{subcaption}

%Esta propuesta debe contener: 


%- Otros antecedentes de relevancia que aporten a la glorificación del trabajo a desarrollar
%-Esta propuesta debe ser entregada a su patrocinante y éste, en señal de acuerdo y aceptación de la misma, la debe enviar a la jefatura de carrera.


\begin{document}
%portada
\begin{titlepage}
	\begin{figure}
		
		\begin{subfigure}[b]{0.5\textwidth}
			\includegraphics[scale=0.45]{figures/diicc.png}
		\end{subfigure}
		\hfill
		\begin{subfigure}[b]{0.1\textwidth}
			\includegraphics[scale=0.4]{figures/escudo_udec.png}
		\end{subfigure}
	\end{figure}
	
	\centering	
	\par\vspace{1cm}
	{\scshape\LARGE Univarsidad de Concepción \par}
	{\scshape Ingeniería civil informática \par}
	\vspace{1cm}
	{\scshape\Large Propuesta memoria de título\par}
	\vspace{1.5cm}
	{\huge\bfseries Diseño e implementacióm de dispositivo (y software) medidor de temperatura para compost\par}
	\vspace{2cm}
	{\Large\itshape Francisco Flores Mellado\par}
	\vfill
	prof. patrocinante\par
	Dr.~Pedro \textsc{Pinacho}

	\vfill

% Bottom of the page
	{\large \today\par}
\end{titlepage}

\section{Introducción}
%contexto del trabajo
%descripción del problema
%forma general de solución propuesta con resultados esperados
Actualmente, el país presenta un creciente desarrollo de la actividad del compostaje como una alternativa a la gestión de residuos orgánicos, los cuales provienen principalmente de restos de alimentos de mercado o ferias libres y de vegetales producto de las podas de parques y jardines. \\
El compost se produce a base de residuos orgánicos y específicamente suele ser utilizado como mejorador de algunas propiedades físicas del suelo como son su estructura, drenaje, aireación, retencion de agua y nutrientes, prevención de la erosión del suelo, recuperación de suelos degradados y superficies alteradas sin uso agrícola. El compostaje se presenta como una alternativa a la quema agrícola.\\
Con el fin de mantener un estándar en la producción, mantenimiento, almacenaje, transporte y posterior venta de compost, es que se creó una norma que busca, en términos generales, promover la gestión adecuada de los recursos sólidos orgánicos generados en el territorio nacional, evitar la producción de plagas, junto con promover y fomentar el desarrollo de la industria nacional de compost. En ese sentido, la norma cubre aspectos relacionados a la clasificación del compost, requisito de materias primas, requisitos del producto compostado, los que incluyen requisitos sanitarios y físico químicos. Estos últimos abarcan desde contenido de nutrientes, capacidad de rehidratación, pH, materia orgánica, hasta olores y humedad.\\
Según la NCh 2880 es de vital importancia el registro y control de temperaturas en las pilas de compost debido a que





\section{Objetivos generales}

\section{Objetivos específicos}

\section{Tareas}

\section{Recursos a utilizar}

\section{Planificación propuesta (tiempo y recursos)}
%considere como plazo final aproximado 31 de marzo de 2021, con defensa y calificación registrada.


\end{document}